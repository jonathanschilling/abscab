\section{Outlook}
\label{sec:outlook}
An obvious extension of this work is to include gradients
of the computed magnetic vector potential and magnetic field.
This issue is already discussed in Ref.~\cite{walstrom_2017}
for the case of the circular wire loop.
%
Precise methods to compute the field gradients would be useful for magnetic field-line tracing applications~\cite{bozhenkov_2013}.
Typical applications in experimental physics do not require the full 16 digits of precision
provided by a \texttt{binary64} implementation of these methods.
An implementation in \texttt{binary32} arithmetic
of the methods presented in this work
is expected to achieve a similar fraction of precision,
since no constants dependent on machine-precision are used in the implementation.
This would correspond to approximately 7 digits of precision
that could be expected from a \texttt{binary32} implementation.
Entry-level graphics processing units (GPUs) (and in particular more advanced models)
can perform parallel computations in \texttt{binary32} much quicker than regular processors of a PC.
The methods presented in this work as well as the gradient methods therefore should be ported to \texttt{binary32}
and would allow for rapid evaluation at guaranteed error levels still low enough for practical applications in experimental physics.
%
Furthermore, gradient-based optimization methods applied to, e.g., finding the coils for a Stellarator device~\cite{zhu_2017}
require calculations of the field gradients.
It is noted that derivatives with respect to the position of the current carrier
are easily expressed via derivatives of the fields.
The change of the field with changing current carrier position is complementary
to the change of field with changing evaluation position (relative to the current carrier).
%
Other types of current carriers are also desirable to have implemented in this framework.
In particular, the case of a filamentary arc segment (the non-closed counterpart of the circular wire loop)
would be of interest as well as finite-extent current carriers.
It is noted that the methods presented by Urankar for this case~\cite{urankar_1,urankar_2,urankar_3,urankar_4,urankar_5}
have recently been reconsidered and improved~\cite{maurer_2020}.
