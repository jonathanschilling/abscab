\section{Introduction}
\label{sec:introduction}
A common task in computational physics is to compute the magnetic field and magnetic vector potential
originating from current-carrying wire arrangements.
These current carriers are often approximated for computational simplicity as infinitely thin filaments
following the center lines of the real current carrier.
A current carrier path specified by ($x$,$y$,$z$) coordinates
can be modeled as a set of straight wire segments from point to point along the polygon describing the current carrier geometry.
Closed circular wire loops are also commonly used as a proxy for a physical coil with helical windings.
Computational methods are needed to compute the magnetic field and the magnetic vector potential
of a single straight wire segment or a single circular wire loop
in order to model more complex current carrier arrangements
by superposition of the current carrier primitives.
Analytical expressions are derived in this work to accurately compute
the magnetic field and the magnetic vector potential of a straight wire segment
and a circular wire loop. The expressions consist of several special cases,
which are switched between depending on the location of the evaluation location
in the coordinate system of the current carrier primitive.
This approach allows to select the most accurate formulation
for a given evaluation location and to make explicit use of simplifications
in certain special cases for speed and accuracy.
The provided implementations have not been optimized for speed.

The magnetic vector field is denoted by $\mathbf{H}$ and
the magnetic flux density $\mathbf{B}$ is then given by $\mathbf{B} = \mu_0 \mu_\mathrm{r} \mathbf{H}$,
where $\mu_0$ is the vacuum magnetic permeability
and $\mu_r$ is the relative permeability, taking material properties into account.
Generally, in the field of plasma physics and in particular in this work,
these two terms are frequently used synonymously
due to the vanishing magnetic susceptibility $|\chi| \ll 1$ of the plasma,
leading to $\mu_r = 1+\chi \approx 1$.
This is equivalent to considering only vacuum magnetic fields.
In this case, the magnetic field and magnetic flux density only differ by a factor of $\mu_0$.

The paper is structured as follows.
In section \ref{sec:methods}, the main numerical method of this work is presented
together with arbitrary-precision reference calculations to benchmark the numerical method.
The results of the comparison between the finite-precision method
and the arbitrary-precision reference data are presented in Sec.~\ref{sec:results}
and then further discussed in Sec.~\ref{sec:discussion}.
An outlook is given in Sec.~\ref{sec:outlook}.
The full derivations of the equations presented in this work are given in~\ref{apx:derivations}.
