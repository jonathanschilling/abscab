\section{Introduction}
\label{sec:introduction}
A common task in computational physics is to compute the magnetic field and magnetic vector potential
originating from current-carrying wire arrangements in the form of coils and cables.
These current carriers can be approximated for computational simplicity as infinitely thin filaments
following the center lines of the real current carrier.
A current carrier path specified by a list of points along the path
can be modeled as a set of straight wire segments from point to point along the polygon describing the current carrier geometry.
Closed circular wire loops appear as well in practise and can be chosen to model, e.g.,
individual windings in a circular coil.
Computational methods are thus needed to compute the magnetic field and the magnetic vector potential
of a single straight wire segment or a single circular wire loop
in order to model more complex current carrier arrangements
by superposition of the current carrier primitives.
Analytical expressions are derived in this work to accurately compute
the magnetic field and the magnetic vector potential of a straight wire segment
and a circular wire loop. The expressions consist of several special cases,
which are switched between depending on the location of the evaluation location
in the coordinate system of the current carrier primitive.
This approach allows to select the most accurate formulation
for a given evaluation location and make explicit use of simplications
in certain special cases for the sake of simplicity, speed and accuracy.
Particular attention was paid to make sure that the expressions presented in this work
obey the correct asymptotic behavior for evaluation locations
far away from and close to the current carriers.
For most test cases, the full relative accuracy of the floating point
arithmetic chosen for implementation is retained throughout the computations
(16 digits of precision in the IEEE754 \texttt{binary64}~\cite{ieee754} implementation),
with few exceptions where accuracy drops by up to 2 digits of precision
(in the \texttt{binary64} case).
The provided implementations have not been optimized for speed.

The magnetic field is denoted by $\mathbf{H}$ and
the magnetic flux density $\mathbf{B}$ is then given by $\mathbf{B} = \mu_0 \mu_\mathrm{r} \mathbf{H}$,
where $\mu_0$ is the vacuum magnetic permeability
and $\mu_r$ is the relative permeability, taking material properties into account.
In the field of plasma physics generally and in this work in particular,
these two terms are frequently used synonymously
due to the vanishing magnetic susceptibility $|\chi| \ll 1$ of the plasma,
leading to $\mu_r = 1+\chi \approx 1$.
This is equivalent to considering only vacuum magnetic fields.
Then, magnetic field and magnetic flux density only differ by a factor of $\mu_0$.
