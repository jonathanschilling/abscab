\section{Derivation of General Formulations}
\label{apx:derivation_of_general_formulations}
The derivations of the starting point formulas presented in Sec.~\ref{sec:methods} are given here.

\subsection{Straight Wire Segment}
The following geometric quantities with $\mathbf{x}_f \equiv \mathbf{x}_{i+1}$ are defined to ease the rest of the derivation:
\begin{align}
 L                   & \equiv | \mathbf{x}_f - \mathbf{x}_i | \, , \\
 \hat{\mathbf{e}}    & \equiv \left(\mathbf{x}_f - \mathbf{x}_i\right) / L \, , \\
 \mathbf{R}_i        & \equiv \mathbf{x} - \mathbf{x}_i \, , \\
 \mathbf{R}_f        & \equiv \mathbf{x} - \mathbf{x}_f \, , \\
 R_i                 & \equiv | \mathbf{R}_i | = | \mathbf{x} - \mathbf{x}_i | \, , \\
 R_f                 & \equiv | \mathbf{R}_f | = | \mathbf{x} - \mathbf{x}_f | \, , \\
 R_{i ||}            & \equiv \hat{\mathbf{e}} \cdot \mathbf{R}_i \, , \\
 R_{f ||}            & \equiv \hat{\mathbf{e}} \cdot \mathbf{R}_f \, , \\
 \mathbf{R}_\perp    & \equiv \mathbf{R}_i - R_{i ||} \hat{\mathbf{e}} \, , \\
 R_\perp             & \equiv | \mathbf{R}_\perp | \quad \mathrm{and} \\
 \mathbf{c}(\lambda) & \equiv \mathbf{x}_i + \lambda \left(\mathbf{x}_f - \mathbf{x}_i\right) \quad \mathrm{for} \quad 0 \leq \lambda \leq 1 \, .
\end{align}
The following relations are also needed:
\begin{align}
       L             & = R_{i ||} - R_{f ||} \\
       R_i^2 - R_f^2 & = L \left( R_{i ||} + R_{f ||} \right) \\
\Rightarrow R_{i ||} & = \frac{R_i^2 - R_f^2}{2 L} + \frac{L}{2} \\
\Rightarrow R_{f ||} & = \frac{R_i^2 - R_f^2}{2 L} - \frac{L}{2}
\end{align}

\subsubsection{Magnetic Vector Potential}
The law of Biot and Savart for the magnetic vector potential of a current density distribution $\mathbf{j}(\mathbf{x})$ is as follows~\cite{jackson}:
\begin{equation}
 \mathbf{A}(\mathbf{x}) = \frac{\mu_0}{4 \pi} \int \frac{\mathbf{j}(\mathbf{x}')}{|\mathbf{x} - \mathbf{x}'|} \mathrm{d}\mathbf{x}' \, .
\end{equation}
The parametrization of points on the line segment $\mathbf{c}(\lambda)$ can be used to apply this to the given geometry of a wire segment:
\begin{align}
 \mathbf{A}(\mathbf{x}) & = \frac{\mu_0 I}{4 \pi} L \hat{\mathbf{e}} \int\limits_0^1 \frac{\mathrm{d}\lambda}{|\mathbf{x} - \mathbf{c}(\lambda)|} \\
        ~               & = \frac{\mu_0 I}{4 \pi} L \hat{\mathbf{e}} \int\limits_0^1 \frac{\mathrm{d}\lambda}{|\mathbf{x} - \mathbf{x}_i - \lambda L \hat{\mathbf{e}}|} \, .
\end{align}
A little bit of geometric intuition is needed to simplify the denominator of the integral:
\begin{align}
 \mathbf{x} - \mathbf{x}_i - \lambda L \hat{\mathbf{e}}
   & = \mathbf{R}_i - \lambda L \hat{\mathbf{e}} \\
 ~ & = \mathbf{R}_i - R_{i ||} \hat{\mathbf{e}} + R_{i ||} \hat{\mathbf{e}} - \lambda L \hat{\mathbf{e}} \\
 ~ & = \mathbf{R}_i - R_{i ||} \hat{\mathbf{e}} + \left( R_{i ||} - \lambda L \right) \hat{\mathbf{e}} \\
 ~ & = \mathbf{R}_\perp + \left( R_{i ||} - \lambda L \right) \hat{\mathbf{e}} \, .
\end{align}
Note that, in particular, $\mathbf{R}_\perp \perp \hat{\mathbf{e}}$ and thus (since $|\hat{\mathbf{e}}|$ = 1) due to Pythagoras:
\begin{equation}
 | \mathbf{x} - \mathbf{x}_i - \lambda L \hat{\mathbf{e}} |^2 = R_\perp^2 + \left( R_{i ||} - \lambda L \right)^2
\end{equation}
and finally with $R_\perp^2 = R_i^2 - R_{i ||}^2$ (also due to Pythagoras):
\begin{align}
 | \mathbf{x} - \mathbf{x}_i - \lambda L \hat{\mathbf{e}} |^2
   & = R_i^2 - R_{i ||}^2 + R_{i ||}^2 - 2 \lambda L R_{i ||} + \lambda^2 L^2 \\
 ~ & = R_i^2 - 2 \lambda L R_{i ||} + \lambda^2 L^2 \, .
\end{align}
It follows:
\begin{equation}
 \mathbf{A}(\mathbf{x})
 = \frac{\mu_0 I}{4 \pi} L \hat{\mathbf{e}} \int\limits_0^1 \frac{\mathrm{d}\lambda}{\sqrt{R_i^2 - 2 \lambda L R_{i ||} + \lambda^2 L^2}} \, . \label{eqn:A_integral}
\end{equation}
For $X = a x^2 + b x + c$ with $a>0$ the following relation holds~\cite{bronstein}:
\begin{equation}
 \int \frac{\mathrm{d}x}{\sqrt{X}} = \frac{1}{\sqrt{a}} \log \left( 2 \sqrt{a X} + 2 a x + b \right) \, .
\end{equation}
Here, $x = \lambda$, $a = L^2$, $b=-2 L R_{i ||}$ and $c=R_i^2$.
The corresponding antiderivative of the integrand in \eqn{A_integral} is:
\begin{align}
 \int&\frac{\mathrm{d}\lambda}{\sqrt{R_i^2 - 2 \lambda L R_{i ||} + \lambda^2 L^2}} \nonumber \\
 =&\, \frac{1}{L} \log \left( 2 \sqrt{L^2\left( L^2 \lambda^2 - 2 L R_{i ||} \lambda + R_i^2 \right)} + 2 L^2 \lambda - 2 L R_{i ||} \right) \, .
\end{align}
The definite integral in \eqn{A_integral} is therefore solved by the following expression:
\begin{align}
 ~ & \int\limits_0^1 \frac{\mathrm{d}\lambda}{\sqrt{R_i^2 - 2 \lambda L R_{i ||} + \lambda^2 L^2}} \\
 = & \frac{1}{L} \Biggl[\phantom{-}\, \log \left( 2 \sqrt{L^2\left( L^2 - 2 L R_{i ||} + R_i^2 \right)} + 2 L^2 - 2 L R_{i ||} \right) \nonumber \\
 ~ & \phantom{\frac{1}{L} \Biggl[} - \log \left( 2 \sqrt{L^2 R_i^2 } - 2 L R_{i ||} \right) \Biggr] \\
 = & \frac{1}{L} \log \left( \frac{ \bcancel{2 L} \sqrt{L^2 - 2 L R_{i ||} + R_i^2} + \bcancel{2} L^{\bcancel{2}} - \bcancel{2 L} R_{i ||} }{ \bcancel{2 L} R_i - \bcancel{2 L} R_{i ||} } \right)
\end{align}
Note that
\begin{align}
                             L^2 & = L (R_{i ||} - R_{f ||} ) \\
                              ~  & = L R_{i ||} - L R_{f ||} \\
\Rightarrow -2 L R_{i ||} + L^2  & = - \bcancel{2} L R_{i ||}  + \bcancel{L R_{i ||}} - L R_{f ||} \\
                              ~  & = -L (R_{i ||} + R_{f ||} ) \\
                              ~  & = R_f^2 - R_i^2 \\
\Rightarrow                R_f^2 & = R_i^2 -2 L R_{i ||} + L^2 \, .
\end{align}
Therefore:
\begin{equation}
 \int\limits_0^1 \frac{\mathrm{d}\lambda}{\sqrt{R_i^2 - 2 \lambda L R_{i ||} + \lambda^2 L^2}}
 = \frac{1}{L} \log \left( \frac{ R_f - R_{f ||} }{ R_i - R_{i  \,\mathrm{d}\varphi||} } \right) \, .
\end{equation}
Inserting this into \eqn{A_integral} leads to the first intermediate result:
\begin{equation}
   \mathbf{A}(\mathbf{x})
 = \frac{\mu_0 I}{4 \pi} \bcancel{L} \bcancel{\frac{1}{L}} \log \left( \frac{ R_f - R_{f ||} }{ R_i - R_{i ||} } \right) \hat{\mathbf{e}}
 = \frac{\mu_0 I}{4 \pi}                                   \log \left( \frac{ R_f - R_{f ||} }{ R_i - R_{i ||} } \right) \hat{\mathbf{e}} \, . \label{eqn:A_first}
\end{equation}
However, if the point $\mathbf{x}$ is located on the line extension of the wire segment, $R_i = R_{i ||}$ and $R_f = R_{f ||}$,
which leads to a $0/0$ division if this formula is directly evaluated.
The solution is to cancel the singular term $(L + R_f - R_i)$, which is also zero for points on the line extension of the wire segment,
in the numerator and the denominator of \eqn{A_first}.
A second look resolves this:
\begin{align}
\frac{ R_f - R_{f ||} }{ R_i - R_{i ||} }
 = & \frac{ 2 L \left( R_f - R_{f ||} \right) }{ 2 L \left( R_i - R_{i ||} \right) }
 =   \frac{ 2 L R_f - 2 L \left( \frac{R_i^2 - R_f^2}{2 L} - \frac{L}{2} \right) }{ 2 L R_i - 2 L \left( \frac{R_i^2 - R_f^2}{2 L} + \frac{L}{2} \right) } \\
 = & \frac{ 2 L R_f - R_i^2 + R_f^2 + L^2 }{ 2 L R_i - R_i^2 + R_f^2 - L^2 } \\
 = & \frac{ 2 L R_f - R_i^2 + R_f^2 + L^2 + L R_i - L R_i + R_i R_f - R_i R_f}{ 2 L R_i - R_i^2 + R_f^2 - L^2 + L R_f - L R_f + R_i R_f - R_i R_f } \\
 = & \frac{\bcancel{(L + R_f - R_i)}(R_f + R_i + L)}{\bcancel{(L + R_f - R_i)}(R_f + R_i - L)}
 =   \frac{R_f + R_i + L}{R_f + R_i - L} \, .
\end{align}
It follows for the vector potential expression:
\begin{equation}
 \mathbf{A}(\mathbf{x}) = \frac{\mu_0 I}{4 \pi} \log \left( \frac{R_f + R_i + L}{R_f + R_i - L} \right) \hat{\mathbf{e}} \, . \label{eqn:A_second}
\end{equation}
The authors of Ref.~\cite{hanson_hirshman_2002} suggest to normalize the length of the wire segment:
\begin{equation}
 \frac{R_f + R_i + L}{R_f + R_i - L} = \frac{1 + \epsilon}{1 - \epsilon} \quad \mathrm{with} ~ \epsilon \equiv \frac{L}{R_i + R_f} \, ,
\end{equation}
leading to
\begin{equation}
 \mathbf{A}(\mathbf{x}) = \frac{\mu_0 I}{4 \pi} \log\left(\frac{1 + \epsilon}{1 - \epsilon} \right) \hat{\mathbf{e}} \, . \label{eqn:A_log_eps}
\end{equation}
This is the result for the magnetic vector potential of a filamentary wire segment presented in Ref.~\cite{hanson_hirshman_2002}.
Note that
\begin{equation}
 \mathrm{artanh}\left( \epsilon \right) = \frac{1}{2} \log\left(\frac{1 + \epsilon}{1 - \epsilon} \right) \, ,
\end{equation}
leading to
\begin{equation}
 \boxed{\mathbf{A}(\mathbf{x}) = \frac{\mu_0 I}{2 \pi} \, \mathrm{artanh} \left( \epsilon \right) \hat{\mathbf{e}}} \, . \label{eqn:A_artanh}
\end{equation}

\subsubsection{Magnetic Field}
The law of Biot and Savart for the magnetic field of a current density distribution $\mathbf{j}(\mathbf{x})$ is as follows~\cite{jackson}:
\begin{equation}
 \mathbf{B}(\mathbf{x}) = \frac{\mu_0}{4 \pi} \int \mathbf{j}(\mathbf{x}') \times \frac{\mathbf{x} - \mathbf{x}'~}{|\mathbf{x} - \mathbf{x}'|^3} \mathrm{d}\mathbf{x}' \, .
\end{equation}
The magnetic field $\mathbf{B}(\mathbf{x})$ is computed from $\mathbf{B} = \nabla \times \mathbf{A}$, applied to \eqn{A_log_eps}.
Define
\begin{equation}
 f(\epsilon) \equiv \log\left(\frac{1 + \epsilon}{1 - \epsilon} \right)
\end{equation}
and it follows:
\begin{equation}
  \frac{4 \pi}{\mu_0 I} \mathbf{B}
 = \nabla \times \left( f(\epsilon) \hat{\mathbf{e}} \right)
 = \nabla f(\epsilon) \times \hat{\mathbf{e}} + f(\epsilon) \underbrace{\nabla \times \hat{\mathbf{e}}}_{=0}
 = f'(\epsilon) \nabla \epsilon \times \hat{\mathbf{e}} \, .
\end{equation}
Note that
\begin{align}
   \nabla \epsilon
 =&\, \nabla \left( \frac{L}{R_i + R_f} \right)
 = \frac{-L}{(R_i + R_f)^2}\left( \nabla R_i + \nabla R_f \right) \nonumber \\
 =&\, \frac{-L}{(R_i + R_f)^2}\left( \frac{\mathbf{R}_i}{R_i} + \frac{\mathbf{R}_f}{R_f} \right) \, .
\end{align}
It follows:
\begin{align}
   \frac{4 \pi}{\mu_0 I} \mathbf{B}
 = & f'(\epsilon) \frac{-L}{(R_i + R_f)^2} \left( \frac{\mathbf{R}_i}{R_i} + \frac{\mathbf{R}_f}{R_f} \right) \times \hat{\mathbf{e}} \\
 = & f'(\epsilon) \frac{L}{(R_i + R_f)^2} \, \hat{\mathbf{e}} \times \left( \frac{\mathbf{R}_i}{R_i} + \frac{\mathbf{R}_f}{R_f} \right) \\
 = & f'(\epsilon) \frac{\epsilon^2}{L}    \, \hat{\mathbf{e}} \times \left( \frac{\mathbf{R}_i}{R_i} + \frac{\mathbf{R}_f}{R_f} \right) \, . \label{eqn:B_intermediate}
\end{align}
Also:
\begin{align}
   \frac{\mathbf{R}_i}{R_i} + \frac{\mathbf{R}_f}{R_f}
 =&\, \frac{\mathbf{R}_i}{R_i} + \frac{\mathbf{R}_i - L \hat{\mathbf{e}} }{R_f}
 =   \frac{R_f \mathbf{R}_i + R_i (\mathbf{R}_i - L \hat{\mathbf{e}}) }{R_i R_f} \nonumber \\
 =&\,   \frac{(R_f+R_i) \mathbf{R}_i + R_i L \hat{\mathbf{e}} }{R_i R_f}
 = \frac{R_f+R_i}{R_i R_f} \mathbf{R}_i + \frac{R_i L}{R_i R_f} \, \hat{\mathbf{e}}
\end{align}
and therefore:
\begin{equation}
   \hat{\mathbf{e}} \times \left( \frac{\mathbf{R}_i}{R_i} + \frac{\mathbf{R}_f}{R_f} \right)
 = \hat{\mathbf{e}} \times \left( \frac{R_f+R_i}{R_i R_f} \mathbf{R}_i + \frac{R_i L}{R_i R_f} \, \hat{\mathbf{e}} \right)
 = \frac{R_f+R_i}{R_i R_f} \, \hat{\mathbf{e}} \times \mathbf{R}_i \, ,
\end{equation}
since $\hat{\mathbf{e}} \times \hat{\mathbf{e}} = 0$.
Inserting this into \eqn{B_intermediate} leads to:
\begin{equation}
   \frac{4 \pi}{\mu_0 I} \mathbf{B}
 = f'(\epsilon) \frac{\epsilon^{\bcancel{2}}}{\bcancel{L}} \, \frac{\bcancel{R_f+R_i}}{R_i R_f} \, \hat{\mathbf{e}} \times \mathbf{R}_i
 = f'(\epsilon) \frac{\epsilon}{R_i R_f} \, \hat{\mathbf{e}} \times \mathbf{R}_i \label{eqn:B_intermediate_2}
\end{equation}
Next, look at $f'(\epsilon)$:
\begin{equation}
   f'(\epsilon)
 = \frac{\bcancel{1 - \epsilon}}{1 + \epsilon} \cdot \frac{1 (1-\epsilon) - (1+\epsilon) (-1)}{(1 - \epsilon)^{\bcancel{2}}}
 = \frac{1 - \epsilon + 1 + \epsilon}{(1 + \epsilon)(1 - \epsilon)}
 = \frac{2}{1 - \epsilon^2}
\end{equation}
and insert this into \eqn{B_intermediate_2}:
\begin{align}
   \frac{4 \pi}{\mu_0 I} \mathbf{B}
 = & \frac{2 \epsilon}{1 - \epsilon^2} \cdot \frac{1}{R_i R_f} \, \hat{\mathbf{e}} \times \mathbf{R}_i \\
 = & \frac{2 L}{\bcancel{R_i + R_f}} \cdot \frac{(R_i + R_f)^{\bcancel{2}}}{(R_i + R_f)^2 - L^2} \cdot \frac{1}{R_i R_f} \, \hat{\mathbf{e}} \times \mathbf{R}_i \, .
\end{align}
This results in the final expression for the magnetic field:
\begin{equation}
 \boxed{\mathbf{B} (\mathbf{x}) = \frac{\mu_0 I}{4 \pi} \frac{2 L (R_i + R_f)}{R_i R_f} \frac{1}{(R_i + R_f)^2 - L^2} \, \hat{\mathbf{e}} \times \mathbf{R}_i } \, .
\end{equation}

\subsection{Circular Wire Loop}
The current density of the wire loop can be expressed as follows:
\begin{equation}
  \mathbf{j}(\mathbf{x}') = I \delta(\rho' - a) \delta(z') \,\hat{\mathbf{e}}_{\varphi'} \, .
\end{equation}

\subsubsection{Magnetic Vector Potential}
The Biot-Savart law for the magnetic vector potential reads:
\begin{align}
  \mathbf{A}(\mathbf{x}) &= \frac{\mu_0  }{4 \pi}
                            \int_{\realnumbers^3}
                              \frac{\mathbf{j}(\mathbf{x}')}{|\mathbf{x} - \mathbf{x}'|} \,\mathrm{d}^3\mathbf{x}' \nonumber \\
             ~           &= \frac{\mu_0 I}{4 \pi}
                            \int_{\realnumbers^3}
                              \frac{\delta(\rho' - a) \delta(z')}{|\mathbf{x} - \mathbf{x}'|} \,\hat{\mathbf{e}}_{\varphi'}
                              \,\mathrm{d}^3\mathbf{x}' \nonumber \\
             ~           &= \frac{\mu_0 I}{4 \pi}
                            \int\limits_{-\infty}^{\infty} \int\limits_{0}^{2 \pi} \int\limits_{0}^{\infty}
                              \frac{\delta(\rho' - a) \delta(z')}{|\mathbf{x} - \mathbf{x}'|} \,\hat{\mathbf{e}}_{\varphi'}
                              \rho' \,\mathrm{d} \rho' \,\mathrm{d} \varphi'  \,\mathrm{d} z' \label{eqn:vecpot_loop_general}
\end{align}
where a change of variables from Cartesian coordinates to cylindrical coordinates was performed in the integral.
The differential volume element was adjusted according to
$\,\mathrm{d}^3\mathbf{x}' = \rho' \,\mathrm{d} \rho' \,\mathrm{d} \varphi'  \,\mathrm{d} z'$.
The coordinate system is rotated around the $z$ axis to yield $\varphi=0$ for the evaluation location $\mathbf{x}$
which is generally acceptable due to the rotational symmetry of the circular wire loop.
Then, $\mathbf{x} = (x, y, z)$ in Cartesian coordinates with
\begin{align}
  x &= \rho \cos(\varphi) = \rho \\
  y &= \rho \sin(\varphi) = 0    \, .
\end{align}
The distance $|\mathbf{x} - \mathbf{x}'|$ is then:
\begin{align}
  |\mathbf{x} - \mathbf{x}'| &= \sqrt{(\rho - \rho' \cos(\varphi'))^2 + \rho'^2 \sin^2(\varphi') + (z - z')^2} \nonumber \\
              ~              &= \sqrt{ \rho^2 + \rho'^2 (\cos(\varphi'))^2 + \sin^2(\varphi')) - 2 \rho \rho' \cos(\varphi') + (z - z')^2} \nonumber \\
              ~              &= \sqrt{ \rho^2 + \rho'^2 + (z - z')^2 - 2 \rho \rho' \cos(\varphi')} \, .
\end{align}
Inserting this into \eqn{vecpot_loop_general} leads to:
\begin{equation}
  \mathbf{A}(\mathbf{x}) = \frac{\mu_0 I}{4 \pi} \int\limits_{-\infty}^{\infty} \int\limits_{0}^{2 \pi} \int\limits_{0}^{\infty}
                                                             \frac{\delta(\rho' - a) \delta(z')}{\sqrt{ \rho^2 + \rho'^2 + (z - z')^2 - 2 \rho \rho' \cos(\varphi')}}
                                                             \,\hat{\mathbf{e}}_{\varphi'}
                                                             \rho' \,\mathrm{d} \rho' \,\mathrm{d} \varphi'  \,\mathrm{d} z'
\end{equation}
and the integrals over $\rho'$ and $z'$ can be evaluated already:
\begin{equation}
  \mathbf{A}(\mathbf{x}) = \frac{\mu_0 I a}{4 \pi}
                           \int\limits_{0}^{2 \pi}
                             \frac{\hat{\mathbf{e}}_{\varphi'} \,\mathrm{d} \varphi'}{\sqrt{ \rho^2 + a^2 + z^2 - 2 \rho a \cos(\varphi')}} \, . \label{eqn:vecpot_loop_phiprime}
\end{equation}
The cylindrical components of the magnetic vector potential~$\mathbf{A}$ are obtained
by dotting above result with the cylindrical unit vector at the evaluation location~$\mathbf{x}$:
\begin{equation}
  \mathbf{A}(\mathbf{x}) =   A_\rho    \,\hat{\mathbf{e}}_\rho
                           + A_\varphi \,\hat{\mathbf{e}}_\varphi
                           + A_z       \,\hat{\mathbf{e}}_z
\end{equation}
with
\begin{align}
  A_\rho    &= \mathbf{A}(\mathbf{x}) \cdot \,\hat{\mathbf{e}}_\rho    \\
  A_\varphi &= \mathbf{A}(\mathbf{x}) \cdot \,\hat{\mathbf{e}}_\varphi \\
  A_z       &= \mathbf{A}(\mathbf{x}) \cdot \,\hat{\mathbf{e}}_z       \, .
\end{align}
The dot products of the cylindrical unit vectors are:
\begin{align}
  \hat{\mathbf{e}}_{\varphi'} \cdot \,\hat{\mathbf{e}}_\rho    &= \begin{pmatrix} -\sin(\varphi') \\ \cos(\varphi') \\ 0 \end{pmatrix}
                                                                  \cdot
                                                                  \begin{pmatrix}  \cos(\varphi ) \\ \sin(\varphi ) \\ 0 \end{pmatrix}
                                                                = \begin{pmatrix} -\sin(\varphi') \\ \cos(\varphi') \\ 0 \end{pmatrix}
                                                                  \cdot
                                                                  \begin{pmatrix}  1 \\ 0 \\ 0 \end{pmatrix}
                                                                = -\sin(\varphi') \\
  \hat{\mathbf{e}}_{\varphi'} \cdot \,\hat{\mathbf{e}}_\varphi &= \begin{pmatrix} -\sin(\varphi') \\ \cos(\varphi') \\ 0 \end{pmatrix}
                                                                  \cdot
                                                                  \begin{pmatrix} -\sin(\varphi ) \\ \cos(\varphi ) \\ 0 \end{pmatrix}
                                                                = \begin{pmatrix} -\sin(\varphi') \\ \cos(\varphi') \\ 0 \end{pmatrix}
                                                                  \cdot
                                                                  \begin{pmatrix}  0 \\ 1 \\ 0 \end{pmatrix}
                                                                =  \cos(\varphi') \\
  \hat{\mathbf{e}}_{\varphi'} \cdot \,\hat{\mathbf{e}}_z       &= \begin{pmatrix} -\sin(\varphi') \\ \cos(\varphi') \\ 0 \end{pmatrix}
                                                                  \cdot
                                                                  \begin{pmatrix} 0 \\ 0 \\ 1 \end{pmatrix}
                                                                = 0 \, . \label{eqn:e_phiprime_dot_ez}
\end{align}
The expression from \eqn{vecpot_loop_phiprime} is inserted into above expressions.
The vertical component~$A_z$ vanishes trivially since the unit vectors are orthogonal,
as can be seen from \eqn{e_phiprime_dot_ez}.
For the radial component $A_\rho$ it follows:
\begin{equation}
  A_\rho = \frac{\mu_0 I a}{4 \pi}
           \int\limits_{0}^{2 \pi}
             \frac{-\sin(\varphi') \,\mathrm{d} \varphi'}{\sqrt{ \rho^2 + a^2 + z^2 - 2 \rho a \cos(\varphi')}} = 0 \, ,
\end{equation}
since the integrand is an odd function of $\varphi'$.
The tangential component $A_\varphi$ is non-zero because the integrand is an even function of $\varphi'$.
It is given by:
\begin{equation}
  A_\varphi(\rho, z) = \frac{\mu_0 I a}{4 \pi}
                       \int\limits_{0}^{2 \pi}
                         \frac{\cos(\varphi') \,\mathrm{d} \varphi'}{\sqrt{ \rho^2 + a^2 + z^2 - 2 \rho a \cos(\varphi')}} \, , \label{eqn:a_phi_general}
\end{equation}
leading to
\begin{equation}
  \mathbf{A}(\mathbf{x}) = A_\varphi(\rho, z) \,\hat{\mathbf{e}}_\varphi \, . \label{eqn:a_cwl_components}
\end{equation}
A change of variables from $\varphi'$ to $\beta$ is performed in order to solve \eqn{a_phi_general}:
\begin{equation}
 \varphi' = 2 \beta + \pi
\end{equation}
which implies
\begin{align}
 \frac{\mathrm{d}\varphi'}{\mathrm{d}\beta} = 2     &\Rightarrow \mathrm{d}\varphi' = 2 \mathrm{d}\beta \\
                                 \varphi'_0 = 0     &\Rightarrow           \beta_0 = - \frac{\pi}{2}   \\
                                 \varphi'_1 = 2 \pi &\Rightarrow           \beta_1 =   \frac{\pi}{2}   \, .
\end{align}
It follows for \eqn{a_phi_general}:
\begin{equation}
 A_\varphi(\rho, z) = \frac{\mu_0 I a}{4 \pi}
                       \int\limits_{-\pi/2}^{\pi/2}
                         \frac{2 \cos(2 \beta + \pi) \,\mathrm{d}\beta}
                              {\sqrt{\rho^2 + a^2 + z^2 - 2 \rho a \cos(2 \beta + \pi)}} \, .
\end{equation}
Note that $\cos(2 \beta + \pi) = - \cos(2 \beta)$:
\begin{equation}
 A_\varphi(\rho, z) = \frac{\mu_0 I a}{2 \pi}
                       \int\limits_{-\pi/2}^{\pi/2}
                         \frac{-\cos(2 \beta) \,\mathrm{d}\beta}
                              {\sqrt{\rho^2 + a^2 + z^2 + 2 \rho a \cos(2 \beta)}} \, . \label{eqn:a_phi_progress}
\end{equation}
In the numerator of the integrand it follows:
\begin{equation}
 -\cos(2 \beta) = -\left(\cos^2(\beta) - \sin^2(\beta) \right) = \sin^2(\beta) - \cos^2(\beta) \, .
\end{equation}
The denominator of the integrand can be reformulated by introducing normalized coordinates
$\rho' = \rho/a$ and $z' = z/a$ as follows:
\begin{align}
 ~ & \rho^2 + a^2 + z^2 + 2 \rho a \cos(2 \beta) \nonumber \\
 ~ &= a^2 \left[z'^2 +      \rho'^2 + 1        + 2 \rho' \cos(2 \beta)         \right] \nonumber \\
 ~ &= a^2 \left[z'^2 + (1 + \rho')^2 - 2 \rho' + 2 \rho' \cos(2 \beta)         \right] \nonumber \\
 ~ &= a^2 \left[z'^2 + (1 + \rho')^2 - 2 \rho' \left(1 - \cos(2 \beta) \right) \right] \nonumber \\
 ~ &= a^2 \left[z'^2 + (1 + \rho')^2 - 2 \rho' \left(1 + \sin^2(\beta) - \cos^2(\beta) \right) \right] \nonumber \\
 ~ &= a^2 \left[z'^2 + (1 + \rho')^2 - 2 \rho' \left(\bcancel{\cos^2(\beta)} + \sin^2(\beta) + \sin^2(\beta) \bcancel{- \cos^2(\beta)} \right) \right] \nonumber \\
 ~ &= a^2 \left[z'^2 + (1 + \rho')^2 - 4 \rho' \sin^2(\beta) \right] \nonumber \\
 ~ &= a^2 \left( z'^2 + (1 + \rho')^2 \right) \Biggl[1 - \underbrace{\frac{4 \rho'}{z'^2 + (1 + \rho')^2}}_{\equiv k^2} \sin^2(\beta) \Biggr] \nonumber \\
 ~ &= a^2 \left( z'^2 + (1 + \rho')^2 \right) \left [1 - k^2 \sin^2(\beta) \right] \label{eqn:a_phi_denom_refactor_start}
\end{align}
with
\begin{equation}
 k^2 = \frac{4 \rho'}{z'^2 + (1 + \rho')^2} \, . \label{eqn:my_k_sq}
\end{equation}
Inserting this into \eqn{a_phi_progress} leads to:
\begin{equation}
 A_\varphi(\rho', z') = \frac{\mu_0 I}
                           {2 \pi}
                      \frac{1}
                           {\sqrt{z'^2 + (1 + \rho')^2}}
                      \int\limits_{-\pi/2}^{\pi/2}
                        \frac{\sin^2(\beta) - \cos^2(\beta)}
                             {\sqrt{1 - k^2 \sin^2(\beta)}}
                        \,\mathrm{d}\beta \, .
\end{equation}
Focusing again on the denominator of the integrand:
\begin{align}
 1 - k^2 \sin^2(\beta)
   &= \cos^2(\beta) + \sin^2(\beta) - \frac{4 \rho'}{z'^2 + (1 + \rho')^2} \sin^2(\beta) \nonumber \\
 ~ &= \cos^2(\beta) + \left(1 - \frac{4 \rho'}{z'^2 + (1 + \rho')^2}\right) \sin^2(\beta) \nonumber \\
 ~ &= \cos^2(\beta) + \frac{z'^2 + (1 + \rho')^2 - 4 \rho'}{z'^2 + (1 + \rho')^2} \sin^2(\beta) \nonumber \\
 ~ &= \cos^2(\beta) + \underbrace{\frac{z'^2 + (1 - \rho')^2}{z'^2 + (1 + \rho')^2}}_{\equiv k_c^2} \sin^2(\beta) \nonumber \\
 ~ &= \cos^2(\beta) + k_c^2 \sin^2(\beta) \label{eqn:a_phi_denom_refactor_done}
\end{align}
with
\begin{equation}
 k_c^2 = \frac{z'^2 + (1 - \rho')^2}{z'^2 + (1 + \rho')^2} \, , \label{eqn:k_c_final}
\end{equation}
leading to:
\begin{equation}
 A_\varphi(\rho', z') = \frac{\mu_0 I}
                           {2 \pi}
                      \frac{1}
                           {\sqrt{z'^2 + (1 + \rho')^2}}
                      \int\limits_{-\pi/2}^{\pi/2}
                        \frac{\sin^2(\beta) - \cos^2(\beta)}
                             {\sqrt{\cos^2(\beta) + k_c^2 \sin^2(\beta)}}
                        \,\mathrm{d}\beta \, .
\end{equation}
The integrand is symmetric about $0$ and therefore the integration domain can be halved if a factor of $2$ is included:
\begin{equation}
 A_\varphi(\rho', z') = \frac{\mu_0 I}
                           {\pi}
                      \frac{1}
                           {\sqrt{z'^2 + (1 + \rho')^2}}
                      \int\limits_{0}^{\pi/2}
                        \frac{\sin^2(\beta) - \cos^2(\beta)}
                             {\sqrt{\cos^2(\beta) + k_c^2 \sin^2(\beta)}}
                        \,\mathrm{d}\beta \, .
\end{equation}
The remaining integral is a complete elliptic integral which can be expressed using the form
introduced by Bulirsch~\cite{bulirsch_3}:
\begin{equation}
  \mathrm{cel}(k_c, p, a, b)
= \int\limits_{0}^{\pi/2}
   \frac{a \cos^2(\varphi) + b \sin^2(\varphi)}
        {  \cos^2(\varphi) + p \sin^2(\varphi)}
   \frac{\mathrm{d}\varphi}
        {\sqrt{\cos^2(\varphi) + k_\mathrm{c}^2 \sin^2(\varphi)}} \, .
\end{equation}
Note that the parameter $a$ of $\mathrm{cel}(k_c, p, a, b)$ is not to be confused with the radius of the wire loop.
A numerical implementation of the general complete elliptic integral $\mathrm{cel}(k_c, p, a, b)$ is provided in the cited article.
The use of this particular implementation is inspired by Ref.~\cite{teal}.
Putting above results together, we arrive at the following expression for $A_\varphi$:
\begin{equation}
 \boxed{A_\varphi(\rho', z') = \frac{\mu_0 I}{\pi}
                               \frac{1}{\sqrt{z'^2 + (1 + \rho')^2}}
                               \,\mathrm{cel}(k_c, 1, -1, 1)} \, . \label{eqn:A_phi_final}
\end{equation}
In Eqn.~(5.37) of Ref.~\cite{jackson} the tangential component is given by:
\begin{align}
  A_\varphi(r, \theta) &= \frac{\mu_0}{4 \pi}
                          \frac{4 I a}{\sqrt{a^2 + r^2 + 2 a r \sin(\theta)}}
                          \left[
                            \frac{(2 - k^2)K(k) - 2 E(k)}{k^2}
                          \right] \label{eqn:aphi_initial}
\end{align}
with
\begin{equation}
  k^2 = \frac{4 a r \sin(\theta)}{a^2 + r^2 + 2 a r \sin(\theta)} \, .
\end{equation}
Here, $K(k)$ and $E(k)$ are the complete elliptic integrals of the first and second kind, respectively.
Spherical coordinates are used with $r \sin(\theta) = \rho$ and $r^2 = \rho^2 + z^2$.
In order to bring this to the form in \eqn{A_phi_final},
an expression for the linear combination of $K(k)$ and $E(k)$ from Ref.~\cite{bulirsch_3} is used:
\begin{equation}
  \lambda K(k) + \mu E(k) = \,\mathrm{cel}(k_c, 1, \lambda + \mu, \lambda + \mu k_c^2)
\end{equation}
where
\begin{equation}
  k^2 + k_\mathrm{c}^2 = 1 \, .
\end{equation}
The argument of the elliptic integrals is considered first:
\begin{align}
  k^2 &= \frac{4 a r \sin(\theta)}{a^2 + r^2 + 2 a r \sin(\theta)}
       = \frac{4 a \rho}{a^2 + r^2 + 2 a \rho}
       = \frac{4 \bcancel{a} \rho}{a^{\bcancel{2}} \left(1 + \frac{r^2}{a^2} + 2 \frac{\rho}{a} \right)} \nonumber \\
  ~   &= \frac{4 \rho'}{1 + \frac{r^2}{a^2} + 2 \rho'}
       = 4 \rho' \left( 1 + \frac{\rho^2 + z^2}{a^2} + 2 \rho' \right)^{-1}
       = 4 \rho' \left( 1 + \rho'^{2} + z'^{2} + 2 \rho' \right)^{-1} \nonumber \\
  ~   &= \frac{4 \rho'}{z'^2 + (1 + \rho')^2} \, .
\end{align}
Thus, $k^2$ from Ref.~\cite{jackson} is equivalent to $k^2$ in \eqn{my_k_sq}.
Inserting this into \eqn{aphi_initial} leads to:
\begin{align}
  A_\varphi(r, \theta) &= \frac{\mu_0 I}{\bcancel{2} \pi}
                          \frac{\bcancel{2}}{\sqrt{z'^2 + (1 + \rho')^2}}
                          \left[
                            \frac{(2 - k^2)K(k) - 2 E(k)}{k^2}
                          \right] \nonumber \\
           ~           &= \frac{\mu_0 I}{\pi}
                          \frac{1}{\sqrt{z'^2 + (1 + \rho')^2}}
                          \left[
                            \frac{(2 - k^2)K(k) - 2 E(k)}{k^2}
                          \right] \, .
\end{align}
The coefficients of the elliptic integrals are given as follows:
\begin{align}
  \lambda &= \frac{2 - k^2}{k^2} = \frac{2}{k^2} - 1 \\
  \mu     &= -\frac{2}{k^2}
\end{align}
and their combinations are as follows:
\begin{align}
  \lambda + \mu       &= \frac{2}{k^2} - 1 - \frac{2}{k^2}     = -1 \\
  \lambda + \mu k_c^2 &= \frac{2}{k^2} - 1 - \frac{2}{k^2} (1 - k^2) \nonumber \\
          ~           &= \frac{2}{k^2} - 1 - \frac{2}{k^2} + 2 =  1 \, .
\end{align}
Putting above results together, we arrive at the following expression for $A_\varphi$:
\begin{equation}
 \boxed{A_\varphi(r, \theta) = \frac{\mu_0 I}{\pi}
                            \frac{1}{\sqrt{z'^2 + (1 + \rho')^2}} \,\mathrm{cel}(k_c, 1, -1, 1)}
\end{equation}
with $\rho' = r/a \sin(\theta)$, $z' = \sqrt{r^2 - \rho^2}/a$ and $k_c$ given by \eqn{k_c_final}).
It is favorable for numerical evaluation of $A_\varphi$ to use the form given in \eqn{A_phi_final}
where the linear combination of the complete elliptic integrals is embedded in the parameters of $\mathrm{cel}(k_c, p, a, b)$
and no precautions need to be taken to deal with cancellations in \eqn{aphi_initial}.

\subsubsection{Magnetic Field}
The magnetic field is computed using $\mathbf{B} = \nabla \times \mathbf{A}$.
In cylindrical coordinates with the form of the magnetic vector potential from \eqn{a_cwl_components}
the curl is given as follows:
\begin{equation}
  \mathbf{B}(\mathbf{x})
= B_\rho \hat{\mathbf{e}}_\rho + B_z \hat{\mathbf{e}}_z
\end{equation}
with
\begin{align}
  B_\rho &= -                \frac{\partial       A_\varphi }{\partial z   }  \label{eqn:b_rho_start} \\
  B_z    &=   \frac{1}{\rho} \frac{\partial (\rho A_\varphi)}{\partial \rho}
          = \frac{A_\varphi}{\rho} + \frac{\partial A_\varphi}{\partial \rho} \label{eqn:b_z_start} \, .
\end{align}
Starting from \eqn{a_phi_general} it is noted that the partial derivatives only act on the denominator of the integrand.
Therefore, consider first:
\begin{align}
  \frac{\partial}{\partial z}& \left( a^2 + z^2 + \rho^2 - 2 a \rho \cos(\varphi) \right)^{-\frac{1}{2}} \nonumber \\
  ~ &= - \bcancel{\frac{1}{2}} \left( a^2 + z^2 + \rho^2 - 2 a \rho \cos(\varphi) \right)^{-\frac{3}{2}} \left( \bcancel{2} z \right) \nonumber \\
  ~ &= \frac{- z}{\left[ a^2 + z^2 + \rho^2 - 2 a \rho \cos(\varphi) \right]^{\frac{3}{2}}} \\
  \frac{\partial}{\partial \rho}& \left( a^2 + z^2 + \rho^2 - 2 a \rho \cos(\varphi) \right)^{-\frac{1}{2}} \nonumber \\
  ~ &= - \bcancel{\frac{1}{2}} \left( a^2 + z^2 + \rho^2 - 2 a \rho \cos(\varphi) \right)^{-\frac{3}{2}} \left( \bcancel{2} \rho - \bcancel{2} a \cos(\varphi) \right) \nonumber \\
  ~ &= \frac{-(\rho - a \cos(\varphi))}{\left[ a^2 + z^2 + \rho^2 - 2 a \rho \cos(\varphi) \right]^{\frac{3}{2}}} \, .
\end{align}
These expressions are used to formulate \eqn{b_rho_start} and
a change of variables from $\varphi$ to $\beta$ analogously to the step from \eqn{a_phi_general} to \eqn{a_phi_progress} is performed.
Finally, normalized coordinates are introduced in the integrand similar to the steps from \eqn{a_phi_denom_refactor_start} to \eqn{a_phi_denom_refactor_done}:
\begin{align}
  B_\rho &= \frac{\partial}{\partial z} \left(
             \frac{\mu_0 I a}{4 \pi}
             \int\limits_{0}^{2 \pi}
               \frac{\cos(\varphi') \,\mathrm{d} \varphi'}{\sqrt{ \rho^2 + a^2 + z^2 - 2 \rho a \cos(\varphi')}} \right) \nonumber \\
    ~    &= - \frac{\mu_0 I a z}{4 \pi}
              \int\limits_{0}^{2 \pi}
                \frac{\cos(\varphi') \,\mathrm{d} \varphi'}
                     {\left[a^2 + z^2 + \rho^2 - 2 a \rho \cos(\varphi') \right]^{\frac{3}{2}}} \nonumber \\
    ~    &=   \frac{\mu_0 I a z}{2 \pi}
              \int\limits_{-\pi/2}^{\pi/2}
                \frac{\cos(2 \beta) \,\mathrm{d}\beta}
                     {\left[a^2 + z^2 + \rho^2 + 2 a \rho \cos(2 \beta) \right]^{\frac{3}{2}}} \nonumber \\
     ~    &=  \frac{\mu_0 I}{2 \pi}
              \frac{\cancel{a^2} z'}{a^{\bcancel{3}} \left( z'^2 + (1 + \rho')^2 \right)^{\frac{3}{2}}}
              \int\limits_{-\pi/2}^{\pi/2}
                \frac{\sin^2(\beta) - \cos^2(\beta)}
                     { \left[\cos^2(\beta) + k_c^2 \sin^2(\beta) \right]^{\frac{3}{2}} }
                \,\mathrm{d}\beta \label{eqn:B_rho_almost}
\end{align}
This integrand is also symmetric about $0$, which can be used to express \eqn{B_rho_almost} in terms of the general complete elliptic integral:
\begin{equation}
  \boxed{B_\rho(\rho', z') =
    \frac{\mu_0 I}{\pi a}
    \frac{z'}{\left[ z'^2 + (1 + \rho')^2 \right]^{\frac{3}{2}}}
    \,\mathrm{cel}(k_c, k_c^2, -1, 1) } \, . \label{eqn:B_rho_final}
\end{equation}
Regarding $B_z$ the first term in \eqn{b_z_start} can already be written down using $\rho = a \rho'$ and $A_\varphi$ from \eqn{A_phi_final}:
\begin{equation}
  B_z = \frac{\mu_0 I}{\pi a}
        \frac{1}{\rho' \sqrt{z'^2 + (1 + \rho')^2}}
        \,\mathrm{cel}(k_c, 1, -1, 1)
        + \frac{\partial A_\varphi}{\partial \rho} \label{eqn:B_z_intermediate}
\end{equation}
The second term is considered next:
\begin{align}
      \frac{\partial A_\varphi}{\partial \rho}
 =&\, \frac{\partial A_\varphi}{\partial \rho'} \frac{\partial \rho'}{\partial \rho}
 =    \frac{1}{a} \frac{\partial A_\varphi}{\partial \rho'} \nonumber \\
 =&\, \frac{1}{a} \frac{\partial}{\partial \rho'} \left[
                               \frac{\mu_0 I}{\pi}
                               \frac{1}{\sqrt{z'^2 + (1 + \rho')^2}}
                               \,\mathrm{cel}(k_c, 1, -1, 1) \right] \nonumber \\
 =&\, \frac{\mu_0 I}{\pi a} \frac{\partial}{\partial \rho'} \left[
        \frac{1}{\sqrt{z'^2 + (1 + \rho')^2}} \,\mathrm{cel}(k_c, 1, -1, 1) \right] \nonumber \\
 =&\,          \frac{\mu_0 I}{\pi a} \Biggl\{ \phantom{+}
        \frac{\partial}{\partial \rho'} \left[ \frac{1}{\sqrt{z'^2 + (1 + \rho')^2}} \right] \,\mathrm{cel}(k_c, 1, -1, 1) \nonumber \\
 ~&\, \phantom{\frac{\mu_0 I}{\pi a} \Biggl\{}         +
        \frac{1}{\sqrt{z'^2 + (1 + \rho')^2}} \frac{\partial}{\partial \rho'} \,\mathrm{cel}(k_c, 1, -1, 1) \Biggr\} \nonumber \\
 =&\,          \frac{\mu_0 I}{\pi a} \Biggl\{ \phantom{+}
        \frac{-(1+\rho')}{\left[z'^2 + (1 + \rho')^2\right]^{3/2}}\,\mathrm{cel}(k_c, 1, -1, 1) \nonumber \\
 ~&\, \phantom{\frac{\mu_0 I}{\pi a} \Biggl\{}         +
        \frac{1}{\sqrt{z'^2 + (1 + \rho')^2}} \frac{\partial (k_c^2)}{\partial \rho'}
        \frac{\partial}{\partial (k_c^2)} \,\mathrm{cel}(k_c, 1, -1, 1)\, \Biggr\} \label{eqn:dAphiDrho} \, .
\end{align}
Consider the remaining derivatives:
\begin{align}
      \frac{\partial (k_c^2)}{\partial \rho'}
 =&\, \frac{\partial}{\partial \rho'} \left( \frac{z'^2 + (1 - \rho')^2}{z'^2 + (1 + \rho')^2} \right) \nonumber \\
 =&\, \frac{1}{z'^2 + (1 + \rho')^2} \left(-2(1 - \rho') - k_c^2 \, 2 (1 + \rho') \right) \nonumber \\
 =&\, \frac{-2}{z'^2 + (1 + \rho')^2} \left(1 - \rho' + k_c^2 (1 + \rho') \right) \nonumber \\
 =&\, \frac{-2}{z'^2 + (1 + \rho')^2} \left(1 + k_c^2 - (1 - k_c^2) \rho' \right)
\end{align}
as well as:
\begin{align}
 \frac{\partial}{\partial (k_c^2)}& \,\mathrm{cel}(k_c, 1, -1, 1) \nonumber \\
 =&\, \frac{\partial}{\partial (k_c^2)} \int\limits_0^{\pi/2} \frac{\sin^2(\varphi) - \cos^2(\varphi)}
                                                                   {\sqrt{\cos^2(\varphi) + k_c^2 \sin^2(\varphi)}} \,\mathrm{d}\varphi \nonumber \\
 =&\, \int\limits_0^{\pi/2} \left[ \sin^2(\varphi) - \cos^2(\varphi) \right]
        \frac{\partial}{\partial (k_c^2)} \left[ \cos^2(\varphi) + k_c^2 \sin^2(\varphi) \right]^{-1/2} \,\mathrm{d}\varphi \nonumber \\
 =&\, \int\limits_0^{\pi/2} \left[ \sin^2(\varphi) - \cos^2(\varphi) \right]
        \left(-\frac{1}{2}\right) \left[ \cos^2(\varphi) + k_c^2 \sin^2(\varphi) \right]^{-3/2} \sin^2(\varphi) \,\mathrm{d}\varphi \nonumber \\
 =&\, -\frac{1}{2} \int\limits_0^{\pi/2} \frac{\left[\sin^2(\varphi) - \cos^2(\varphi) \right] \sin^2(\varphi)}
                                              {\left[\cos^2(\varphi) + k_c^2 \sin^2(\varphi) \right]^{3/2}}  \,\mathrm{d}\varphi \, .
\end{align}
These terms are now inserted into \eqn{dAphiDrho}:
\begin{align}
      \frac{\partial A_\varphi}{\partial \rho}
 =&\, \frac{\mu_0 I}{\pi a} \Biggl\{ \phantom{+}
        \frac{-(1+\rho')}{\left[z'^2 + (1 + \rho')^2\right]^{3/2}} \,\mathrm{cel}(k_c, 1, -1, 1) \nonumber \\
 ~&\, \phantom{\frac{\mu_0 I}{\pi a} \Biggl\{}         +
        \frac{1}{\sqrt{z'^2 + (1 + \rho')^2}}
        \frac{\bcancel{-2}}{z'^2 + (1 + \rho')^2} \left(1 + k_c^2 - (1 - k_c^2) \rho' \right) \nonumber \\
 ~&\, \phantom{\frac{\mu_0 I}{\pi a} \Biggl\{ +}
        \bcancel{\left(-\frac{1}{2} \right)}
        \int\limits_0^{\pi/2}
          \frac{\left[\sin^2(\varphi) - \cos^2(\varphi) \right] \sin^2(\varphi)}
               {\left[\cos^2(\varphi) + k_c^2 \sin^2(\varphi) \right]^{3/2}}  \,\mathrm{d}\varphi \, \Biggr\} \nonumber \\
 =&\, \frac{\mu_0 I}{\pi a} \Biggl\{
      \phantom{+}\,
      \frac{-(1 + \rho')}{\left[z'^2 + (1 + \rho')^2\right]^{3/2}}
      \,\mathrm{cel}(k_c, 1, -1, 1) \nonumber \\
 ~&\, \phantom{\frac{\mu_0 I}{\pi a} \Biggl\{}
      + \frac{1 + k_c^2 - (1 - k_c^2) \rho'}{\left[z'^2 + (1 + \rho')^2\right]^{3/2}}
        \int\limits_0^{\pi/2}
          \frac{\left[\sin^2(\varphi) - \cos^2(\varphi) \right] \sin^2(\varphi)}
               {\left[\cos^2(\varphi) + k_c^2 \sin^2(\varphi) \right]^{3/2}}  \,\mathrm{d}\varphi \Biggr\} \, .
\end{align}
This form of $\partial A_\varphi / \partial \rho$ is now inserted into \eqn{B_z_intermediate}.
$\partial A_\varphi / \partial \rho$ has to be multiplied by a factor of $\rho'$ in order to factor out a common prefactor $1/\rho'$.
This leads to:
\begin{align}
  B_z
 =&\, \frac{\mu_0 I}{\pi a}
      \frac{1}{\rho' \sqrt{z'^2 + (1 + \rho')^2}} \Biggl\{ \nonumber \\
 ~&\, \,\mathrm{cel}(k_c, 1, -1, 1) - \frac{\rho' (1+\rho')}{z'^2 + (1 + \rho')^2} \,\mathrm{cel}(k_c, 1, -1, 1) \nonumber \\
 ~&\, + \frac{\rho' \left[1 + k_c^2 - (1 - k_c^2) \rho'\right]}{z'^2 + (1 + \rho')^2}
          \int\limits_0^{\pi/2}
          \frac{\left[\sin^2(\varphi) - \cos^2(\varphi) \right] \sin^2(\varphi)}
               {\left[\cos^2(\varphi) + k_c^2 \sin^2(\varphi) \right]^{3/2}}  \,\mathrm{d}\varphi \Biggr\} \nonumber \\
 =&\, \frac{\mu_0 I}{\pi a}
      \frac{1}{\rho' \sqrt{z'^2 + (1 + \rho')^2}} \Biggl\{ \nonumber \\
 ~&\, \left( 1 - \frac{\rho' (1+\rho')}{z'^2 + (1 + \rho')^2}\right) \,\mathrm{cel}(k_c, 1, -1, 1) \nonumber \\
 ~&\, + \frac{\rho' \left[1 + k_c^2 - (1 - k_c^2) \rho'\right]}{z'^2 + (1 + \rho')^2}
          \int\limits_0^{\pi/2}
          \frac{\left[\sin^2(\varphi) - \cos^2(\varphi) \right] \sin^2(\varphi)}
               {\left[\cos^2(\varphi) + k_c^2 \sin^2(\varphi) \right]^{3/2}}  \,\mathrm{d}\varphi \Biggr\} \nonumber \\
 =&\, \frac{\mu_0 I}{\pi a}
      \frac{1}{\rho' \sqrt{z'^2 + (1 + \rho')^2}} \Biggl\{ \nonumber \\
 ~&\, \left( 1 - \frac{\rho' (1+\rho')}{z'^2 + (1 + \rho')^2}\right)
      \int\limits_0^{\pi/2}
        \frac{\sin^2(\varphi) - \cos^2(\varphi)}
             {\sqrt{\cos^2(\varphi) + k_c^2 \sin^2(\varphi)}} \,\mathrm{d}\varphi \nonumber \\
 ~&\, + \frac{\rho' \left[1 + k_c^2 - (1 - k_c^2) \rho'\right]}{z'^2 + (1 + \rho')^2}
          \int\limits_0^{\pi/2}
          \frac{\left[\sin^2(\varphi) - \cos^2(\varphi) \right] \sin^2(\varphi)}
               {\left[\cos^2(\varphi) + k_c^2 \sin^2(\varphi) \right]^{3/2}}  \,\mathrm{d}\varphi \Biggr\} \, .
\end{align}
The integrand of the two integrals can be combined.
A nutritious one needs to be included in the first integrand:
\begin{align}
  B_z
 =&\, \frac{\mu_0 I}{\pi a}
      \frac{1}{\rho' \sqrt{z'^2 + (1 + \rho')^2}}
      \int\limits_0^{\pi/2} \,\mathrm{d}\varphi\,
        \frac{\sin^2(\varphi) - \cos^2(\varphi)}
             {\left[\cos^2(\varphi) + k_c^2 \sin^2(\varphi) \right]^{3/2}} \nonumber \\
 \Biggl\{ &\left( 1 - \frac{\rho' (1+\rho')}{z'^2 + (1 + \rho')^2} \right)
            \left[ \cos^2(\varphi) + k_c^2 \sin^2(\varphi) \right] \nonumber \\
 ~&\,     + \frac{\rho' \left[1 + k_c^2 - (1 - k_c^2) \rho'\right]}{z'^2 + (1 + \rho')^2} \sin^2(\varphi)
        \Biggr\} \, .
\end{align}
Consider for now only the part inside the $\{\}$ of above equation:
\begin{align}
 ~&\,   \left( 1 - \frac{\rho' (1+\rho')}{z'^2 + (1 + \rho')^2} \right)
        \left[ \cos^2(\varphi) + k_c^2 \sin^2(\varphi) \right] \nonumber \\
 ~&\, + \frac{\rho' \left[1 + k_c^2 - (1 - k_c^2) \rho'\right]}{z'^2 + (1 + \rho')^2} \sin^2(\varphi) \nonumber \\
 =&\, \frac{1}{z'^2 + (1 + \rho')^2} \Biggl\{
        \left[ z'^2 + (1 + \rho')^2 - \rho' (1+\rho') \right] \left[ \cos^2(\varphi) + k_c^2 \sin^2(\varphi) \right] \nonumber \\
 ~&\, \phantom{\frac{1}{z'^2 + (1 + \rho')^2} \Biggl\{}
      + \rho' \left[1 + k_c^2 - (1 - k_c^2) \rho'\right] \sin^2(\varphi) \Biggr\}
\end{align}
and in there also only the part inside the $\{\}$:
\begin{align}
 ~&\, \left[ z'^2 + (1 + \rho')^2 - \rho' (1+\rho') \right] \left[ \cos^2(\varphi) + k_c^2 \sin^2(\varphi) \right] \nonumber \\
 ~&\, + \rho' \left[1 + k_c^2 - (1 - k_c^2) \rho'\right] \sin^2(\varphi) \nonumber \\
 =&\, \left[ z'^2 + (1 + \rho')^2 - \rho' (1+\rho') \right] \left[ \cos^2(\varphi) + \frac{z'^2 + (1 - \rho')^2}{z'^2 + (1 + \rho')^2} \sin^2(\varphi) \right] \nonumber \\
 ~&\, + \rho' \left[1 + k_c^2 - (1 - k_c^2) \rho'\right] \sin^2(\varphi) \nonumber \\
 =&\, \left[ z'^2 + (1 + \rho')^2 \right] \cos^2(\varphi) + \left[ z'^2 + (1 - \rho')^2 \right] \sin^2(\varphi)                                      \nonumber \\
 ~&\, - \rho' (1+\rho') \left[ \cos^2(\varphi) + k_c^2 \sin^2(\varphi) \right] + \rho' \left[1 + k_c^2 - (1 - k_c^2) \rho'\right] \sin^2(\varphi) \, . \label{eqn:starStar}
\end{align}
The two contributions are simplified separately:
\begin{align}
 ~&\, \left[ z'^2 + (1 + \rho')^2 \right] \cos^2(\varphi) + \left[ z'^2 + (1 - \rho')^2 \right] \sin^2(\varphi) \nonumber \\
 =&\, \left[ z'^2 + 1 + 2 \rho' + \rho'^2 \right] \cos^2(\varphi) + \left[ z'^2 + 1 - 2 \rho' + \rho'^2  \right] \sin^2(\varphi) \nonumber \\
 =&\, z'^2 + 1 + \rho'^2 + 2 \rho' \left[ \cos^2(\varphi) - \sin^2(\varphi) \right] \label{eqn:intermediate1}
\end{align}
as well as:
\begin{align}
 ~&\, - \rho' (1+\rho') \left[ \cos^2(\varphi) + k_c^2 \sin^2(\varphi) \right] + \rho' \left[1 + k_c^2 - (1 - k_c^2) \rho'\right] \sin^2(\varphi) \nonumber \\
 =&\, - \rho' \cos^2(\varphi) - \rho'^2 \cos^2(\varphi) \cancel{- \rho' k_c^2 \sin^2(\varphi)}                           \bcancel{- \rho'^2 k_c^2 \sin^2(\varphi)} \nonumber \\
 ~&\, + \rho' \sin^2(\varphi)                           \cancel{+ \rho' k_c^2 \sin^2(\varphi)} - \rho'^2 \sin^2(\varphi) \bcancel{+ \rho'^2 k_c^2 \sin^2(\varphi)} \nonumber \\
 =&\, - \rho' \left[ \cos^2(\varphi) - \sin^2(\varphi) \right] - \rho'^2 \underbrace{\left[ \cos^2(\varphi) + \sin^2(\varphi) \right]}_{=1} \label{eqn:intermediate2}
\end{align}
Combining the results from~\eqn{intermediate1} and~\eqn{intermediate2} to form \eqn{starStar} leads to:
\begin{align}
 ~&\, z'^2 + 1 \bcancel{+ \rho'^2} + \cancel{2} \rho' \left[ \cos^2(\varphi) - \sin^2(\varphi) \right] \cancel{- \rho' \left[ \cos^2(\varphi) - \sin^2(\varphi) \right]} \bcancel{- \rho'^2} \nonumber \\
 =&\, z'^2 + 1 + \rho' \left[ \cos^2(\varphi) - \sin^2(\varphi) \right] \, .
\end{align}
The full form of $B_z$ is assembled again now based on the recent findings to remind ourselves of the state of things:
\begin{align}
 B_z
 =&\, \frac{\mu_0 I}{\pi a}
      \frac{1}{\rho' \sqrt{z'^2 + (1 + \rho')^2}} \nonumber \\
 ~&\, \int\limits_0^{\pi/2}
        \frac{\sin^2(\varphi) - \cos^2(\varphi)}
             {\left[\cos^2(\varphi) + k_c^2 \sin^2(\varphi) \right]^{3/2}}
        \frac{z'^2 + 1 + \rho' \left[ \cos^2(\varphi) - \sin^2(\varphi) \right]}{z'^2 + (1 + \rho')^2}
        \,\mathrm{d}\varphi \, . \label{eqn:B_z_core}
\end{align}
The numerator of the second term in above integrand is now restructured :
\begin{align}
 ~&\, z'^2 + 1 + \rho' \left[ \cos^2(\varphi) - \sin^2(\varphi) \right] \nonumber \\
 =&\, \frac{1}{2} z'^2 + \frac{1}{2} \left( z'^2 + 1 - \rho'^2 \right) + \frac{1}{2} \left\{ 1 + \rho'^2 + 2 \rho' \left[ \cos^2(\varphi) - \sin^2(\varphi) \right] \right\} \nonumber \\
 =&\, \phantom{+}~
        \frac{1}{2} z'^2 \left[ \cos^2(\varphi) + \sin^2(\varphi) \right]
      + \frac{1}{2} \left( z'^2 + 1 - \rho'^2 \right) \nonumber \\
 ~&\, + \frac{1}{2} \Bigl[   \underbrace{\left( 1 + 2 \rho' + \rho'^2 \right)}_{=(1+\rho')^2} \cos^2(\varphi)
                           + \underbrace{\left( 1 - 2 \rho' + \rho'^2 \right)}_{=(1-\rho')^2} \sin^2(\varphi) \Bigr] \nonumber \\
 =&\,   \frac{1}{2} \left[ z'^2 + (1 + \rho')^2 \right] \cos^2(\varphi) + \frac{1}{2} \left[ z'^2 + (1 - \rho')^2 \right] \sin^2(\varphi) \nonumber \\
 ~&\, + \underbrace{\frac{1}{2} \left( z'^2 + 1 + \rho'^2 \right)}_{(*)} \underbrace{- \rho'^2}_{(**)} \label{eqn:bZCore} \, .
\end{align}
Consider $(*)$ and $(**)$ separately:
\begin{align}
 (*)
 =&\, \frac{1}{2} z'^2 + \frac{1}{4} \Bigl[ \underbrace{1 + 2 \rho' + \rho'^2}_{=(1+\rho')^2} + \underbrace{1 - 2 \rho' + \rho'^2}_{=(1-\rho')^2} \Bigr] \nonumber \\
 =&\, \frac{1}{4} \left[ z'^2 + (1 + \rho')^2 + z^2 + (1 - \rho')^2 \right] \nonumber \\
 =&\, \frac{1}{4} \left[ z'^2 + (1 + \rho')^2 \right] + \frac{1}{4} \left[ z^2 + (1 - \rho')^2 \right]
\end{align}
and
\begin{align}
  (**)
 =&\, \frac{1}{4} \rho' \left[ z'^2 + 1 - 2 \rho' + \rho'^2 - \left( z'^2 + 1 + 2 \rho' + \rho'^2 \right) \right] \nonumber \\
 =&\, \frac{1}{4} \rho' \left[ z'^2 + ( 1 - \rho')^2 \right] - \frac{1}{4} \rho' \left[ z'^2 + (1 + \rho')^2 \right] \, .
\end{align}
These expressions are now inserted into \eqn{bZCore}, leading to:
\begin{align}
 ~&\, \phantom{+}~
        \frac{1}{2} \left[ z'^2 + (1 + \rho')^2 \right] \cos^2(\varphi) + \frac{1}{2} \left[ z'^2 + (1 - \rho')^2 \right] \sin^2(\varphi) \nonumber \\
 ~&\, + \frac{1}{4} \left[ z'^2 + (1 + \rho')^2 \right] + \frac{1}{4} \left[ z^2 + (1 - \rho')^2 \right] \nonumber \\
 ~&\, + \frac{1}{4} \rho' \left[ z'^2 + ( 1 - \rho')^2 \right] - \frac{1}{4} \rho' \left[ z'^2 + (1 + \rho')^2 \right] \nonumber \\
 =&\, \left[ z'^2 + (1 + \rho')^2 \right] \Biggl\{
        \frac{1}{2} \cos^2(\varphi) + \underbrace{\frac{z'^2 + (1 - \rho')^2}{z'^2 + (1 + \rho')^2}}_{=k_c^2} \sin^2(\varphi) \nonumber \\
 ~&\, \phantom{\left[ z'^2 + (1 + \rho')^2 \right] \Biggl\{}
        + \frac{1}{4} \Biggl[ 1 + \underbrace{\frac{z'^2 + (1 - \rho')^2}{z'^2 + (1 + \rho')^2}}_{=k_c^2}
                              - \Biggl( 1 - \underbrace{\frac{z'^2 + (1 - \rho')^2}{z'^2 + (1 + \rho')^2}}_{=k_c^2} \Biggr) \rho' \Biggr] \Biggr\} \nonumber \\
 =&\, \left[ z'^2 + (1 + \rho')^2 \right] \left\{
          \frac{1}{2} \left[ \cos^2(\varphi) + k_c^2 \sin^2(\varphi) \right]
        + \frac{1}{4} \left[ 1 + k_c^2 - \left( 1 - k_c^2 \right) \rho' \right] \right\} \, .
\end{align}
This is now inserted back into \eqn{B_z_core}:
\begin{align}
  B_z
 =&\, \frac{\mu_0 I}{\pi a}
      \frac{1}{\rho' \sqrt{z'^2 + (1 + \rho')^2}}
      \int\limits_0^{\pi/2}
        \frac{\sin^2(\varphi) - \cos^2(\varphi)}
             {\left[\cos^2(\varphi) + k_c^2 \sin^2(\varphi) \right]^{3/2}} \nonumber \\
 ~&\,   \left\{   \frac{1}{2} \left[ \cos^2(\varphi) + k_c^2 \sin^2(\varphi) \right]
                + \frac{1}{4} \left[ 1 + k_c^2 - \left( 1 - k_c^2 \right) \rho' \right] \right\} \,\mathrm{d}\varphi \, .
\end{align}
The integral can be split up back again into two integrals:
\begin{align}
  B_z
 =&\, \frac{\mu_0 I}{\pi a}
      \frac{1}{\rho' \sqrt{z'^2 + (1 + \rho')^2}} \Biggl[
        \frac{1}{2} \underbrace{\int\limits_0^{\pi/2} \frac{\sin^2(\varphi) - \cos^2(\varphi)}
                                                           {\sqrt{\cos^2(\varphi) + k_c^2 \sin^2(\varphi)}} \,\mathrm{d}\varphi}_{=\textrm{cel}(k_c, 1, -1, 1)} \nonumber \\
 ~&\, + \frac{1}{4} \left[ 1 + k_c^2 - \left( 1 - k_c^2 \right) \rho' \right]
          \underbrace{\int\limits_0^{\pi/2} \frac{\sin^2(\varphi) - \cos^2(\varphi)}
                                                 {\left[\cos^2(\varphi) + k_c^2 \sin^2(\varphi) \right]^{3/2}} \,\mathrm{d}\varphi}_{=\textrm{cel}(k_c, k_c^2, -1, 1)} \Biggr] \, .
\end{align}
This concludes the derivation of the expression for $B_z$ of a circular wire loop
and here is the final result as published in Ref.~\cite{teal}:
\begin{align}
 B_z(\rho', z')
 =&\, \frac{\mu_0 I}{2 \pi a}
   \frac{1}{\rho' \sqrt{z'^2 + (1 + \rho')^2}}
   \Bigl[ \nonumber \\
 ~&\, \textrm{cel}(k_c, 1, -1, 1)
     + \frac{1 + k_c^2 - \left( 1 - k_c^2 \right) \rho'}{2} \textrm{cel}(k_c, k_c^2, -1, 1)
   \Bigr]
\end{align}

\section{Derivation of Special Case Formulations}
\label{apx:derivation_of_special_case_formulations}


The parameter~$\epsilon$ approaches a value of~$1$ as the location of the evaluation point comes closer to the wire segment.
Thus, the denominator in \eqn{A_z_tilde} vanishes, leading to a logarithmic singularity in~$\tilde{A}_z$
as the evaluation point~$(\rho', z')$ gets nearer to the wire segment.
It is therefore favorable to formulate the near-field method in terms of~$(1-\epsilon)$:
\begin{align}
  1-\epsilon =&\, 1 - \frac{1}{r_i + r_f} = \frac{r_i + r_f - 1}{r_i + r_f} = \frac{r_i + r_f - 1}{(r_i + r_f - 1) + 1} = \frac{n}{n + 1}
\end{align}
with~$n \equiv r_i + r_f - 1$ approaching~$0$ as the evaluation point approaches the wire segment.
Inserting this formulation for~$1-\epsilon$ into \eqn{A_z_tilde} leads to:
\begin{align}
  \tilde{A}_{z,\mathrm{nf}} (\rho', z')
  =&\, \frac{1}{2}  \left[ \ln\left(2 - (1-\epsilon)        \right) - \ln \left(1 - \epsilon    \right) \right] \nonumber \\
  =&\, \frac{1}{2}  \left[ \ln\left(2 - \frac{n}{n + 1}     \right) - \ln \left(\frac{n}{n + 1} \right) \right] \nonumber \\
  =&\, \frac{1}{2}  \left[ \ln\left(\frac{2(n+1) - n}{n + 1}\right) - \ln \left(\frac{n}{n + 1} \right) \right] \nonumber \\
  =&\, \frac{1}{2}  \left[ \ln\left(n + 2                   \right) \cancel{- \ln\left(n + 1 \right)} - \ln \left( n \right) \cancel{+ \ln \left( n + 1 \right)} \right] \nonumber \\
  =&\, \frac{1}{2}  \left[ \ln\left(n + 2                   \right) - \ln \left( n \right) \right] \, .
\end{align}




This is now reformulated to use normalized quantities
(as done above for the computation of the magnetic vector potential):
\begin{align}
 B_\varphi
 =&\, \frac{\mu_0 I}{4 \pi}
      \frac{\rho'}{r_i}
      \left(\frac{R_i}{R_f} + 1 \right)
      \frac{2 \bcancel{L}}{L^{\bcancel{2}} \left[ \left(r_i + r_f\right)^2 - 1\right]} \nonumber \\
 =&\, \frac{\mu_0 I}{4 \pi L}
      \left(\frac{\bcancel{r_i}}{\bcancel{r_i} r_f} + \frac{1}{r_i} \right)
      \frac{2 \rho'}{\left(r_i + r_f\right)^2 - 1} \nonumber \\
 =&\, \frac{\mu_0 I}{4 \pi L}
      \left(\frac{1}{r_f} + \frac{1}{r_i} \right)
      \frac{2 \rho'}{\left(r_i + r_f\right)^2 - 1} \, .
\end{align}
