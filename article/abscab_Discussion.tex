\section{Discussion}
\label{sec:discussion}
Accurate methods have been presented in this work
to compute the magnetic vector potential~$\mathbf{A}$
and the vacuum magnetic field~$\mathbf{B}$
of filamentary current carriers
in the form of a straight wire segment
and a circular wire loop.
Particular attention was paid to make sure that the implemented formulas
are capable of achieving nearly full precision of the floating-point arithmetic
in which they are implemented.
%
The methods for the straight wire segment
achieve 15 out of 16 digits of precision for both
the magnetic vector potential and the magnetic field
in the \texttt{binary64} implementation.
%
The methods for the circular wire loop
achieve 15 digits of precision for the magnetic vector potential
and for the magnetic field far away from the wire loop.
Close to the wire loop, accuracy of the magnetic field methods
drops to 14 digits of precision.
%
The reference data was computed using arbitrary-precision arithmetic
in two very different implementations (namely the open source \texttt{mpmath} Python package
and the commercial Mathematica software) and verified to be sufficiently accurate
for testing the \texttt{binary64} implementations.
This reference data is provided along with this article
to allow the readers to test their own routines.
%
An application is presented where these methods
are used to approximate the magnetic field of a circular current loop
by using a current along a polygon aligned with the shape of the wire loop.
It is demonstrated that the approximation converges to the analytical result
of the wire loop down to machine precision and remains there
for further increases in the number of vertices of the approximating polygon.
This further demonstrates the robustness of the methods presented in this work.
It was found that compensated summation techniques were required
in order to correctly accumulate the superposition of a large number
of individual polygon segments.
The selected second-order Kahan-Babushka summation is therefore used by default in the reference implementation.
%
The asymptotic behaviour of the formulas
when evaluating~$\mathbf{A}$ and~$\mathbf{B}$ close to the current carrier
and far away from it is verified to produce the correct results.
%
The free-boundary part of the Variational Moments Equilibrium Code (VMEC)~\cite{hirshman_1986}
uses interpolation of the vacuum magnetic field produced by the confinement coils
of a Stellarator device~\cite{spitzer_1958}.
It is frequently the case that coils overlap with the region
in which the cached interpolation table is computed for further use.
During iterations of VMEC, it can happen that this way
field values close to the coils enter the computation
and it is deemed beneficary for the robustness of the computation
if those values attain the correct asymptotic values.
